% status: 0
% chapter: TBD

\title{GraphQL and Web APIs}

\author{Averill Cate, Jr}
\orcid{1234-5678-9012}
\affiliation{%
  \institution{The University of Indiana}
  \streetaddress{}
  \city{Bloomington} 
  \state{Indiana} 
  \postcode{47408}
}
\email{acate@iu.edu}

\author{Gregor von Laszewski}
\affiliation{%
  \institution{Indiana University}
  \streetaddress{Smith Research Center}
  \city{Bloomington} 
  \state{IN} 
  \postcode{47408}
  \country{USA}}
\email{laszewski@gmail.com}

% The default list of authors is too long for headers}
\renewcommand{\shortauthors}{G. v. Laszewski}

\begin{abstract}
This project is for Advanced Cloud Computing, Indiana University Spring 
Semester 2018.  Through this project, we will demonstrate the use of GraphQL \cite{hid505FacebookGraphQL2018}
and cloud computing services as a means for constructing and delivering web 
services.  The project will rely on The Santa Rita  Experimental Range (SRER) 
website and data sets maintained by the Univeristy of Arizona and second, 
related web site, the Walnut Gulch Experimental Watershed (WGEW) Online Data 
Access, maintained by the United States Department of Agriculure's, Agriculure 
Research Service (ARS).  Neither of these existing sites provide modernized 
data services, yet both sites provide potentiall valuable data to the research 
community.  

\end{abstract}

\keywords{hid-sp18-505, GraphQL, Web, Services}

\maketitle

\section{Introduction and Project Outline}
The Santa Rita Experimental is an experimental range currently maintained by 
the University of Arizona.  The experimental range was the first of its kind 
and was established in 1902 \cite{SrerWebSite2018}.  The SRER website, in its 
current state is not structured to deliver data as a service.

The Walnut Gulch Experimental Watershed is also an outdoor research facility 
located in southestern Arizona \cite{WgewWebSite2018}.  Like the SRER site, 
the WGEW preicpitation and runoff data site is not structured to delivered as 
services.

The general outline for this paper is:

\begin{enumerate}
\item Use web scraping tools to download, parse and clean existing data from each 
site.  
\item Load the data into a database.  
\item Develop a small web application to provide two web services.  The two 
services will deliver the same data.  However, one of the services will use 
GraphQL to deliver data and the other will use REST \cite{}hid505swaggerio2018 to 
deliver its data. 
\item Develop a simple load testing tool that will be used to determine 
performance metrics for both services and compare the results of the load tests.
\item Time permitting, create a simple Amazon Lambda service that will compute some metric related to 
the preciption/runoff data.  The purpose is to demonstrate how cloud services can 
be used to off-load costs and dependencies on a distributed system like the one 
proposed in this project.
\item Time permitting, develop a 3-node Raspberry Pi cluster to host the application.  The purpose 
of this, is to demonstrate how a system like this one can be hosted and deployed 
on hardware infrastructure that is cheap and relatively simple to maintain.
\end{enumerate}

\section{GraphQL}
Content on GraphQL

\section{REST/SWAGGER}
Content on REST/Swagger here.

\section{Data Acquistion}
Describe how the SRER data were scraped, parsed, cleaned and stored.

Describe how the WGEW data were scraped, parsed, cleaned and stored.

\section{Web API Application}
Describe the tools used and how the web api application was developed.
I have done most of this work already.

The source code is currently hosted at: https://github.com/acatejr/eapi.git

\secton{GraphQL / REST/Swagger Comparison}
Describe the tools used and the development of the load testing tools.  
Also, describe the results.  Table summary ,etc.

\section{Amazon Lambda}
If there is time describe and implement the Amazon Lambda that the web 
API/application can use.

\section{Raspberry Pi Cluster}
If there is time, describe and develop a three-node Raspberry Pi cluster and 
deploy the web api application to that cluser.

\section{Conclusion}
The conclusion will go here.

\begin{acks}

  The authors would like to thank Dr.~Gregor~von~Laszewski for his support and 
  suggestions to write this paper.
  
\end{acks}

\bibliographystyle{ACM-Reference-Format}
\bibliography{report} 

https://www.tucson.ars.ag.gov/dap/